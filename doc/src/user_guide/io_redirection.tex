\clearpage
\section{I/O Redirection}\label{sec:io}

Rayleigh writes all text output (e.g., error messages, iteration counter, etc.) to stdout by default.  Different computing centers handle stdout in different ways, but typically one of two path is taken. On some machines, a log file is created immediately and updated continuously as the simulation runs. On other machines, stdout is buffered on-node and written to disk only when the run has terminated.

There are situations where it can be advantageous to have a regularly updated log file whose update frequency may be controlled. This feature exists in Rayleigh and may be accessed by assigning values to \textbf{stdout\_flush\_interval} and \textbf{stdout\_file} in the io controls namelist.
\begin{lstlisting}
&io_controls_namelist
stdout_flush_interval = 1000
stdout_file = `routput'
/
\end{lstlisting}
Set stdout\_file to the name of a file that will contain Rayleigh's text output. In the example above, a file named \textit{routput} will be appear in the simulation directory and will be updated periodically throughout the run. The variable stdout\_flush\_interval determines how many lines of text are buffered before they are flushed to routput. Rayleigh prints time-step information during each time step, and so setting this variable to a relatively large number (e.g., 100+) prevents excessive disk access from occurring throughout the run. In the example above, a text buffer flush will occur once 1000 lines of text have been accumulated.

Changes in the time-step size and self-termination of the run will also force a text-buffer flush. Unexpected crashes and sudden termination by the system job scheduler do not force a buffer flush. Note that the default value of stdout\_file is \textbf{`nofile'}. If this value is specified, output will directed to normal stdout.

