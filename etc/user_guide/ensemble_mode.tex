\clearpage
\section{Ensemble Mode}
Rayleigh can also be used to run multiple simulations under the umbrella of a single executable.  This functionality is particularly useful for running parameter space studies, which often consist of mulitple, similarly-sized simulations, in one shot.  Moreover, as some queuing systems favor large jobs over small jobs, an ensemble mode is useful for advancing multiple small simulations through the queue in a reasonable timeframe. 

 
Running Rayleigh in ensemble mode is relatively straightforward.  To begin with, create a directory for each simulation as you normally would, and place an appropriately modified main\_input into each directory.  These directories should all reside within the same parent directory.  Within that parent directory, you should place a copy of the Rayleigh executable (or a softlink).  In addition, you should create a text file named \textbf{run\_list} that contains the name of each simulation directory, one name per line.  An ensemble job may then be executed by calling Rayleigh with \textbf{nruns} command line flag as:

\begin{lstlisting}
user@machinename ~/runs/ $ mpiexec -np Y ./rayleigh.opt -nruns X
\end{lstlisting}

Here, Y is the total number of cores needed by all X simulations listed in run\_list.
     
\textbf{Example:}
Suppose you wish to run three simulations at once from within a parent directory named \textit{ensemble} and that the simulation directories are named run1, run2, and run3.    When performing an \textit{ls} from within \textit{ensemble}, you should see 5 items.

\begin{lstlisting}
user@machinename ~/runs/ $ cd ensemble
user@machinename ~/runs/ensemble $ ls
rayleigh.opt          run1          run2          run3          run_list
\end{lstlisting}
In this example, the contents of run\_list should be the \textit{local} names of your ensemble run-directories, namely run1, run2, and run3.

\begin{lstlisting}
user@machinename ~runs/ensemble $ more run_list 
run1
run2
run3
          <--  place an empty line here
\end{lstlisting}
Note that some Fortran implementations will not read the last line in run\_list unless it ends in a newline character.  Avoid unexpected crashes by hitting "enter" following your final entry in run\_list.  

Before running Rayleigh, make sure you know how many cores each simulation needs by examining the main\_input files:

\begin{lstlisting}
user@machinename ~runs/ensemble $ head run1/main_input 
&problemsize_namelist
 n_r = 128
 n_theta = 192
 nprow = 16
 npcol = 16
/

user@machinename ~runs/ensemble $ head run2/main_input 
&problemsize_namelist
 n_r = 128
 n_theta = 384
 nprow = 32
 npcol = 16
/

user@machinename ~runs/ensemble $ head run3/main_input 
&problemsize_namelist
 n_r = 64
 n_theta = 192
 nprow = 16
 npcol = 16
/
\end{lstlisting}

In this example, we need a total of 1024 cores (256+512+256) to execute three simulations, and so the relevant call to Rayleigh would be:
\begin{lstlisting}
user@machinename ~/runs/ $ mpiexec -np 1024 ./rayleigh.opt -nruns 3
\end{lstlisting}

\textbf{Closing Notes:}
When running in ensemble mode, it is \textit{strongly recommended} that you redirect standard output for each simulation to a text file (see \S \ref{sec:io}).   Otherwise, all simulations write to the same default (machine-dependent) log file, making it difficult to read.  Moreover, some machines such as NASA Pleiades will terminate a run if the log file becomes too long.  This is easy to do when multiple simulations are writing to the same file.

Finally, The flags -nprow and -npcol \textbf{are ignored} when -nruns is specified.  The row and column configuration for all simulations needs to be specified in their respective main\_input files instead.
